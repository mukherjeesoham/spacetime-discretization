\documentclass[nofootinbib,preprintnumbers,superscriptaddress,notitlepage]{revtex4-1}

\usepackage{times}
\usepackage[usenames]{color}
\usepackage{amsfonts}
\usepackage{amsmath}
\usepackage{amssymb}
\usepackage{bm}
\usepackage{dcolumn}
\usepackage{enumerate}
\usepackage{epsfig}
\usepackage{graphicx}
\usepackage{graphics}
\usepackage[latin1]{inputenc}
\usepackage{latexsym}
\usepackage{rotating}
\usepackage{hyperref}
\usepackage[caption=false]{subfig}


%setup packages
\hypersetup{
    colorlinks=true,
    linkcolor=red,
    filecolor=magenta,      
    urlcolor=magenta,
    citecolor=magenta,
}

%Define equation environment shorthand
\newcommand{\<}{\begin{equation}}
\newcommand{\?}{\end{equation}}
%=========================================================================
\begin{document}
%=========================================================================
\title{Solving the scalar wave equation using spacetime discretization}
\date{\today}
\maketitle

%=========================================================================
\subsection{Introduction}
%=========================================================================

We are interested in finding solutions to the scalar wave equation using a 
finite element approach using spacetime discretization. Specifically, we want
to find solutions $u(x, t)$ to
\begin{equation}
\label{eq:wave_equation}
u_{tt} - c^2 \bigtriangleup u = 0, \qquad \text{in} \quad \Omega \times (0, T],
\end{equation}
with boundary and initial conditions (we will check if these are necessary
and/or sufficient below)
\begin{eqnarray}
\label{eq:boundary_conditions}
u =& 0,\quad \partial_{n} u &= 0 \quad \text{in} \quad \partial\Omega,\\
\label{eq:initial_data}
u(\cdot, 0) =& f,\quad u_t(\cdot, 0) &=  0 \quad \text{in} \quad \Omega,
\end{eqnarray}
where Eq.~(\ref{eq:boundary_conditions}) enforces a reflecting boundary
condition at the spatial boundaries, and Eq.~(\ref{eq:initial_data}) specifies
the initial data in terms of the wave profile and it's derivative at $t=0$.\\

%=========================================================================
\subsection{Variational Formulation}
%=========================================================================

We start by computing the variational formulation of
Eq.~(\ref{eq:wave_equation}). If $u(x,t)$ is a solution to
Eq.~(\ref{eq:wave_equation}), then for any reasonable test function $v \in
\Omega \times (0, T]$
\begin{equation}
u_{tt}\,v - c^2 (\bigtriangleup u)\, v = 0.
\end{equation}
Integrating over the domain $\mathcal{M} = \Omega \times (0, T]$, and using
integration by parts for converting the second order derivatives to first
order derivatives, we get
\begin{eqnarray}
\int\limits_{\mathcal{M}} u_{tt}\,v \mathrm{\,dt\, d\Omega} 
- c^2 \int\limits_{\mathcal{M}} \nabla\cdot(\nabla u)\, v \mathrm{\,dt\, d\Omega} 
&=& 0,\\
\label{eq:intbyparts}
\int\limits_{\Omega}\left[u_t v \right]_{0}^{T}\,\mathrm{d\Omega}  
- \int\limits_{\Omega}\mathrm{d\Omega}\int\limits_{T} u_t v_t \mathrm{\,dt}
+ c^2\int\limits_{T}\,\mathrm{dt}\int\limits_{\Omega} \nabla u \cdot \nabla v \,\mathrm{\,d\Omega}
- c^2\int\limits_{T}\,\mathrm{dt}\int\limits_{\Omega} \nabla \cdot \left(v\,\nabla u\right) \,\mathrm{\,d\Omega}
&=& 0.
\end{eqnarray}
Applying Gauss law to the fourth term in the right hand side of Eq.~(\ref{eq:intbyparts}), we get
\begin{equation}
\label{eq:initialweakform}
\int\limits_{\Omega}\left[u_t v \right]_{0}^{T}\,\mathrm{d\Omega}  
- \int\limits_{\mathcal{M}} u_t v_t \mathrm{\,dtd\Omega}
+ c^2\int\limits_{\mathcal{M}} \nabla u \cdot \nabla v \,\mathrm{\,dt d\Omega}
- c^2\int\limits_{T}\,\mathrm{dt}\oint\limits_{\partial\Omega} v\, \left(\nabla u \cdot \mathbf{n}\right) \,\mathrm{\,d\Gamma}
= 0.
\end{equation}
The above equation is the intial weak form of the scalar wave equation. We now
apply the boundary conditions to set some of the terms to
zero\footnote{\textcolor{red}{How we apply these boundary conditions
explicitly in the code is something I'm still working on.}}. Consider the
fourth term in Eq.~(\ref{eq:initialweakform}). We set this to zero, since our
boundary conditions demand that the flux $(\nabla u \cdot \mathbf{n})$ at the
boundary is zero; see Eq.~(\ref{eq:boundary_conditions}).The first term can
also be (partially) eliminated using using the initial condtions.
\textcolor{red}{However, the term $u_t v |_{T}$ remains, since we do not have
a boundary condition there. We could, however, choose a function $v$, which
vanishes at the final time slice, but I'm not sure if that's allowed. For the
time, we assume that the first term can also be set to zero in
Eq.~(\ref{eq:boundary_conditions})}.\\

Therefore, we have, together with the Dirichlet boundary conditions
\begin{eqnarray}
\label{eq:variationalform}
- \int\limits_{\mathcal{M}} u_t v_t \mathrm{\,dtd\Omega}
+ c^2\int\limits_{\mathcal{M}} \nabla u \cdot \nabla v \,\mathrm{\,dt d\Omega}
&=& 0, \quad \text{in}~\Omega,\\
u &=& 0, \quad \text{in}~\partial\Omega. 
\end{eqnarray}

%=========================================================================
\subsection{Spacetime discretization}
%=========================================================================

For the sake of simplicity, we consider Eq.~(\ref{eq:variationalform}) in 1D,
where Eq.~(\ref{eq:variationalform}) becomes much simpler, and reduces to
\begin{equation}
\label{eq:variationalform_final}
- \int\limits_{\mathcal{M}} u_t v_t \mathrm{\,dt dx}
+ c^2\int\limits_{\mathcal{M}} u_x v_x \,\mathrm{\,dt dx}
= 0.
\end{equation}
The above equation has been derived directly in \cite{Zumbusch}. We now
discretize Eq.~(\ref{eq:variationalform_final}) using a spacetime
discretization approach, and express $u$ in terms of $M$ basis functions in
time $\tilde{\psi}_l(t)$, and $N$ basis functions in space $\tilde{\phi}_i(x)$
as
\begin{equation}
u = \sum\limits_{l=0}^{M}\sum\limits_{i=0}^{N}\tilde{u}_i^l \, \tilde{\phi}_i(x)\, \tilde{\psi}_l(t), \\
\end{equation}
and choose the test function as any of the basis functions
\begin{equation}
v = \tilde{\phi}_j(x)\, \tilde{\psi}_m(t),
\end{equation}
as we are free to choose any (reasonable) function in $\Omega$. Plugging these
into Eq.~(\ref{eq:variationalform_final}), we arrive at (setting $c=1$ for convinience)
\begin{equation}
\label{eq:summation_expanded}
- \sum\limits_{i=0}^{N}\sum\limits_{l=0}^{M} u_i^l 
\int\limits_{x} \mathrm{dx}\, \tilde{\phi}_i(x)\tilde{\phi}_j(x)
\int\limits_{T} \mathrm{dt}\, \partial_t\tilde{\psi}_l(t)\partial_t\tilde{\psi}_m^0(t)
+ \sum\limits_{i=0}^{M}\sum\limits_{l=0}^{N} u_i^l 
\int\limits_{T} \mathrm{dt}\, \tilde{\psi}_l(t)\tilde{\psi}_m(t)
\int\limits_{x} \partial_x\tilde{\phi}_i(x)\partial_x\tilde{\phi}_j(x) = 0.
\end{equation}
Defining the mass matrix M and stiffness matrix A in the spatial dimension,
\begin{eqnarray}
\label{eq:matrix_space}
M := m_{ij} &=& \int\limits_{x} \mathrm{dx}\, \tilde{\phi}_i(x)\tilde{\phi}_j(x), \\
A := a_{ij} &=& \int\limits_{x} \partial_x\tilde{\phi}_i(x)\partial_x\tilde{\phi}_j(x),
\end{eqnarray}
and two additional matrices
\begin{eqnarray}
\label{eq:matrix_time}
k_{lm} &=& \int\limits_{T} \mathrm{dt}\, \tilde{\psi}_l(t)\tilde{\psi}_m(t), \\
p_{lm} &=& \int\limits_{T} \mathrm{dt}\, \partial_t\tilde{\psi}_l(t)\partial_t\tilde{\psi}_m(t),
\end{eqnarray}
we can express Eq.~(\ref{eq:summation_expanded}) as
\begin{equation}
- \sum\limits_{l=0}^{M}\sum\limits_{i=0}^{N} u_i^l\, m_{ij}\, p_{lm} 
+ \sum\limits_{l=0}^{M}\sum\limits_{i=0}^{N} u_i^l\, a_{ik}\, k_{lm} = 0.
\end{equation}
Notice that we have two free indices $(j,m)$ and hence $(j \times m)$
independent equations, the same number as the number of unknowns $(i\times
l)$, so things look consistent till now. \textcolor{red}{However, we have
boundary conditions, and how many degrees of freedom do they take away?}. If
we use hat functions as basis functions for both space and time basis
dimensions Eqs.~(\ref{eq:matrix_space}--\ref{eq:matrix_time}), the integrals
are simple to evaluate. We will discuss the case for $k_{lm}$ and $p_{lm}$,
since it will give us an implicit time-stepping equation. Let
\begin{equation}
\psi_j(t) =  
\begin{cases}
0 & \text{if $t\in \left[t_{l-1},t_{l+1}\right]$} \\
\dfrac{t - t_{l-1}}{t_l - t_{l-1}} & \text{if $t\in \left[t_{l-1},t_{l}\right]$} \\
\dfrac{t_{l+1} - t}{t_{l+1} - t_{l}} & \text{if $t\in \left[t_{l},t_{l+1}\right]$}
\end{cases}.
\end{equation}
Then, we immediately notice that the product of two functions (or their
derivatives) would only be non-zero, if there are in the same element, i.e.
$[t_{l-1}, t_{l+1}]$. Thus, Eq.~(\ref{eq:summation_expanded}) now reduces to
(localized in a single element in time)
\begin{equation}
\label{eq:summation_reduced}
- \sum\limits_{l=0}^{M}\sum\limits_{i=0}^{N} u_i^l\, m_{ij}\, 
\int\limits_{t_{l-1}}^{{t_{l+1}}} \mathrm{dt}\, \partial_t\tilde{\psi}_l(t)\partial_t\tilde{\psi}_m^0(t) 
+ \sum\limits_{l=0}^{M}\sum\limits_{i=0}^{N} u_i^l\, a_{ik}\, 
\int\limits_{t_{l-1}}^{{t_{l+1}}} \mathrm{dt}\, \tilde{\psi}_l(t)\tilde{\psi}_m(t) = 0.
\end{equation}
The only surviving elements in $k_{lm}$ and $p_{lm}$
\begin{eqnarray}
\label{time-stepping}
 k_{l-1, l} = h_0/6, \quad k_{ll} &=& 2 h_0/3, \quad k_{l, l+1} = h_0/6, \\
 p_{l-1, l} = -1/h_0, \quad p_{ll} &=& 2/h_0, \quad p_{l, l+1} = -1/h_0,
\end{eqnarray}
where $h_0 = t_l - t_{l-1}$ is the unform spacing in time. Finally, plugging Eq.~(\ref{time-stepping}) into
Eq.~(\ref{eq:summation_reduced}), gives us an implicit step in time, as
\begin{eqnarray}
- \sum\limits_{i=0}^{N} m_{ij}\, \left(u_i^{l-1} p_{l-1, l} + u_i^{l} p_{l, l} + u_i^{l+1} p_{l, l+1} \right)
+ \sum\limits_{i=0}^{N} a_{ij}\, \left(u_i^{l-1} k_{l-1, l} + u_i^{l} k_{l, l} + u_i^{l+1} k_{l, l+1} \right) = 0, \\
M  \left( \dfrac{1}{h_0}\mathbf{u}^{l+1} -  \dfrac{2}{h_0}\mathbf{u}^{l} + \dfrac{1}{h_0}\mathbf{u}^{l-1} \right)
= - A \left( \dfrac{h_0}{6}\mathbf{u}^{l+1} +  \dfrac{2 h_0}{3}\mathbf{u}^{l} + \dfrac{h_0}{6}\mathbf{u}^{l-1} \right) 
\end{eqnarray}
Therfore, given $\mathbf{u}^r$ and $\mathbf{u}^{r-1}$, both vectors of length $N$, we can solve for $\mathbf{u}^{k+1}$, using the equation
\begin{equation}
M  \left(\mathbf{u}^{r+1} -  2\mathbf{u}^{r} + \mathbf{u}^{r-1} \right)
= - A h_0^2 \left( \dfrac{1}{6}\mathbf{u}^{r+1} +  \dfrac{2}{3}\mathbf{u}^{r} + \dfrac{1}{6}\mathbf{u}^{r-1} \right),
\end{equation}
which can be cast in the form of $\tilde{A}\, x = b$, where

\begin{eqnarray}
\tilde{A} &=& M + \dfrac{h_0^2}{6}A, \\
x &=& \mathbf{u}^{r+1}, \\
b &=& 2\,\mathbf{u}^{r}\left(M - \dfrac{2 h_0^2}{3}A\right) - \mathbf{u}^{r-1} \left(M + \dfrac{h_0^2}{6}A\right)
\end{eqnarray}


\bibliography{notes}
\end{document}